%!TEX root = ./ui.tex

\section{Design Thinking } % (fold)
\label{sec:visual_design}
This section answers the two questions: 
\begin{enumerate}[a)]
	\item \emph{What is design thinking, and}
	\item \emph{Why is it important}
\end{enumerate}


\subsection{What is Design Thinking?} % (fold)
\label{sub:what_is_design_thinking}

[Definition] 


\begin{tcolorbox}[
	width=\textwidth,
	% colback={codegray},
	title={\textbf{A Design Thinking Problem Solving Example}},
	outer arc=0mm,
	arc=0mm,
	boxrule=1pt,
	% coltitle=white
	]    
Thinking outside of the box can provide an innovative solution to a sticky problem. However, thinking outside of the box can be a real challenge as we naturally develop patterns of thinking that are modelled on the repetitive activities and commonly accessed knowledge we surround ourselves with. It takes something to break away from a situation where we’re too closely involved to be able to find better possibilities. 
\par \vspace{0.5em}
To illustrate how a fresh way of thinking can create unexpectedly good solutions, let’s look at a famous story. Some years ago, an incident occurred where a truck driver had tried to pass under a low bridge. Alas, he failed, and the truck became firmly lodged under the bridge. The driver was unable to continue driving through or reverse out.
\par \vspace{0.5em}
The story goes that as the truck became stuck, it caused massive traffic problems, which resulted in emergency personnel, engineers, firefighters, and truck drivers gathering to negotiate various solutions so as to dislodge the truck.
\par \vspace{0.5em}
Emergency workers were debating whether to dismantle parts of the truck or chip away at parts of the bridge. Each spoke of a solution which fitted within his or her respective level of expertise. In the heat of the emergency, all parties carried on with their ways of viewing the problem, including the truck driver, whose initial dismay over a scraped roof had turned into a deeper concern.
\par \vspace{0.5em}
A boy walking by and witnessing the intense debate looked at the truck, at the bridge, then looked at the road and said nonchalantly, \emph{``Why not just let the air out of the tires?''} to the absolute amazement of all the specialists and experts trying to unpick the problem.
\par \vspace{0.5em}
When the solution was tested, the truck was able to drive free with ease, having suffered only the damage caused by its initial attempt to pass underneath the bridge. Whether or not the story actually happened in real life, it symbolizes the struggles we face where oftentimes the most obvious solutions are the ones hardest to come by because of the self-imposed constraints we work within.
\end{tcolorbox}   

Challenging our assumptions and everyday knowledge is often difficult for us humans, as we rely on building patterns of thinking in order not to have to learn everything from scratch every time. We rely on doing everyday processes more or less unconsciously---for example, when we get up in the morning, eat, walk, and read---but also when we assess challenges at work and in our private lives. Especially experts and specialists rely on their solid thought patterns, patterns that serve them well in their respective fields, not to mention the people to whom they deliver their skills. Even so, it can be very challenging and difficult for experts to start questioning their knowledge. Pride aside, it can prove more than a little disconcerting to think that many years of education and practical experience can hinder rather than help in dealing with a problem.


Design thinking is firmly based on generating a \textbf{holistic and emphatic understanding} of the problems that people face, and that it involves ambiguous or inherently subjective concepts such as emotions, needs, motivations, and drivers of behaviors.
This contrasts with a solely scientific approach, where there’s more of a distance in the process of understanding and testing the user’s needs and emotions---e.g., via quantitative research. Tim Brown sums up that design thinking is a third way: design thinking is essentially a problem-solving approach, crystallized in the field of design, which combines a holistic user-centered perspective with rational and analytical research with the goal of creating innovative solutions.

Design thinking is an iterative and non-linear process. This simply means that the design team continuously use their results to review, question, and improve their initial assumptions, understandings and results. Results from the final stage of the initial work process inform our understanding of the problem, help us determine the parameters of the problem, enable us to redefine the problem, and, perhaps most importantly, provide us with new insights so we can see any alternative solutions that might not have been available with our previous level of understanding.

\section{Summary} % (fold)
\label{sec:summary}
Design thinking is essentially a problem-solving approach specific to design, which involves assessing known aspects of a problem and identifying the more ambiguous or peripheral factors that contribute to the conditions of a problem. This contrasts with a more scientific approach where the concrete and known aspects are tested in order to arrive at a solution. Design thinking is an iterative process in which knowledge is constantly being questioned and acquired so it can help us redefine a problem in an attempt to identify alternative strategies and solutions that might not be instantly apparent with our initial level of understanding. Design thinking is often referred to as ‘outside the box thinking’, as designers are attempting to develop new ways of thinking that do not abide by the dominant or more common problem-solving methods – just like artists do. At the heart of design thinking is the intention to improve products by analyzing how users interact with products and investigating the conditions in which they operate. Design thinking offers us a means of digging that bit deeper to uncover ways of improving user experiences.

