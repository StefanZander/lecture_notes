%%%%%%%%%%%%%%%%%%%%%%%%%%%%%%%%%%%%%%%%%%%%%%%%%%%%%%%%%%%%%%%%%%%%%%
% How to use writeLaTeX: 
%
% You edit the source code here on the left, and the preview on the
% right shows you the result within a few seconds.
%
% Bookmark this page and share the URL with your co-authors. They can
% edit at the same time!
%
% You can upload figures, bibliographies, custom classes and
% styles using the files menu.
%
% If you're new to LaTeX, the wikibook is a great place to start:
% http://en.wikibooks.org/wiki/LaTeX
%
%%%%%%%%%%%%%%%%%%%%%%%%%%%%%%%%%%%%%%%%%%%%%%%%%%%%%%%%%%%%%%%%%%%%%%
\documentclass[a4paper, justified, notoc]{tufte-handout} % SZA: remove 'notoc' to regain tufte-style TOC

%\geometry{showframe}% for debugging purposes -- displays the margins

\usepackage{amsmath}

% Set up the images/graphics package
\usepackage{graphicx}
\setkeys{Gin}{width=\linewidth,totalheight=\textheight,keepaspectratio}
\graphicspath{{figures/}}

\title{How to Design Intuitive User Interfaces\thanks{Selected lecture of the course ``Nutzerzentrierte Softwareentwicklung''}}
\author[opt Author]{Prof.\ Dr.\ Stefan Zander}
% \date{16.\ Februar 2018}  % if the \date{} command is left out, the current date will be used

% The following package makes prettier tables.  We're all about the bling!
\usepackage{booktabs}

% The units package provides nice, non-stacked fractions and better spacing
% for units.
\usepackage{units}

% The fancyvrb package lets us customize the formatting of verbatim
% environments.  We use a slightly smaller font.
\usepackage{fancyvrb}[baw]
\fvset{fontsize=\normalsize}

% SZA: Added to use a space between paragraphs
\setlength{\parskip}{0.5em}

% Small sections of multiple columns
\usepackage{multicol}

% Provides paragraphs of dummy text
\usepackage{lipsum}

% These commands are used to pretty-print LaTeX commands
\newcommand{\doccmd}[1]{\texttt{\textbackslash#1}}% command name -- adds backslash automatically
\newcommand{\docopt}[1]{\ensuremath{\langle}\textrm{\textit{#1}}\ensuremath{\rangle}}% optional command argument
\newcommand{\docarg}[1]{\textrm{\textit{#1}}}% (required) command argument
\newenvironment{docspec}{\begin{quote}\noindent}{\end{quote}}% command specification environment
\newcommand{\docenv}[1]{\textsf{#1}}% environment name
\newcommand{\docpkg}[1]{\texttt{#1}}% package name
\newcommand{\doccls}[1]{\texttt{#1}}% document class name
\newcommand{\docclsopt}[1]{\texttt{#1}}% document class option name

% SZA: Allows colored boxes
\usepackage[framemethod=tikz]{mdframed}

\usepackage[utf8]{inputenc} % this is needed for umlauts
% \usepackage[ngerman]{babel} % this is needed for umlauts
\usepackage[T1]{fontenc}    % this is needed for correct output of umlauts in pdf


% SZA: Enables coloured text boxes
\usepackage{tcolorbox}



% SZA: Added to provide learning objectives section
\newenvironment{lernziele}{
	\begin{mdframed}[hidealllines=true,backgroundcolor=gray!20] 
	\small \itshape
	\noindent \underline{Objectives:} 
	} 
	{ 
	\end{mdframed}
}

% SZA: The note commands extends the marginnote with additional text
\newcommand{\note}[1] {
	\marginnote{\textbf{Note:}\quad #1 } }


% SZA: Added to support JavaScript in LaTeX Code
%Define the listing package
\usepackage{listings} %code highlighter
\usepackage{color} %use color
\definecolor{mygreen}{rgb}{0,0.6,0}
\definecolor{mygray}{rgb}{0.5,0.5,0.5}
\definecolor{mymauve}{rgb}{0.58,0,0.82}
\definecolor{codegray}{rgb}{0.82,0.82,0.82}
\definecolor{examplerule}{rgb}{0.23, 0.48, 0.57}
\definecolor{exampleback}{rgb}{0.82, 0.91, 0.98}

 
%Customize a bit the look
\lstset{ %
backgroundcolor=\color{gray!20}, % choose the background color; you must add \usepackage{color} or \usepackage{xcolor}
basicstyle=\small, % the size of the fonts that are used for the code
breakatwhitespace=false, % sets if automatic breaks should only happen at whitespace
breaklines=true, % sets automatic line breaking
captionpos=b, % sets the caption-position to bottom
commentstyle=\color{mygreen}, % comment style
deletekeywords={...}, % if you want to delete keywords from the given language
escapeinside={\%*}{*)}, % if you want to add LaTeX within your code
extendedchars=true, % lets you use non-ASCII characters; for 8-bits encodings only, does not work with UTF-8
frame=single, % adds a frame around the code
keepspaces=true, % keeps spaces in text, useful for keeping indentation of code (possibly needs columns=flexible)
keywordstyle=\color{blue}, % keyword style
% language=Octave, % the language of the code
morekeywords={*,...}, % if you want to add more keywords to the set
numbers=left, % where to put the line-numbers; possible values are (none, left, right)
numbersep=5pt, % how far the line-numbers are from the code
numberstyle=\large\color{black}, % the style that is used for the line-numbers
rulecolor=\color{black}, % if not set, the frame-color may be changed on line-breaks within not-black text (e.g. comments (green here))
showspaces=false, % show spaces everywhere adding particular underscores; it overrides 'showstringspaces'
showstringspaces=false, % underline spaces within strings only
showtabs=false, % show tabs within strings adding particular underscores
stepnumber=1, % the step between two line-numbers. If it's 1, each line will be numbered
stringstyle=\color{mymauve}, % string literal style
tabsize=2, % sets default tabsize to 2 spaces
title=\lstname % show the filename of files included with \lstinputlisting; also try caption instead of title
}
%END of listing package%
 
\definecolor{darkgray}{rgb}{.4,.4,.4}
\definecolor{purple}{rgb}{0.65, 0.12, 0.82}
\definecolor{darkgreen}{rgb}{.0,.5,.0}

 
%define Javascript language
\lstdefinelanguage{JavaScript}{
keywords={typeof, new, true, false, catch, function, return, null, catch, switch, var, if, in, while, do, else, case, break},
keywordstyle=\color{orange}\bfseries,
ndkeywords={class, export, boolean, throw, implements, import, this},
ndkeywordstyle=\color{darkgray}\bfseries,
identifierstyle=\color{black},
sensitive=false,
comment=[l]{//},
morecomment=[s]{/*}{*/},
commentstyle=\color{darkgray}\ttfamily,
stringstyle=\color{blue}\ttfamily,
morestring=[b]',
morestring=[b]"
}
 
\lstset{
language=JavaScript,
extendedchars=true,
basicstyle=\small\ttfamily,
showstringspaces=false,
showspaces=false,
numbers=left,
numberstyle=\tiny,
numbersep=9pt,
tabsize=2,
breaklines=true,
showtabs=true,
captionpos=b,
frame=l % was 'l' || was lines
}


% SZA: Print listing captions to the right margin to better accomodate the tufte style
% from: https://tex.stackexchange.com/questions/282485/use-listings-in-tufte-book-with-captions-in-margin
\makeatletter
% textwidth Tuftian float for listings
\newenvironment{listing}[1][htbp]
  {\ifvmode\else\unskip\fi\begin{@tufte@float}[#1]{lstlisting}{}}
  {\end{@tufte@float} } % SZA: Added \vspace{-2em} \vspace command in order to better align the following paragraph
% fullwidth Tuftian float for listings
\newenvironment{listing*}[1][htbp]%
  {\ifvmode\else\unskip\fi\begin{@tufte@float}[#1]{lstlisting}{star}}
  {\end{@tufte@float}}
% enable re-use of \listoflistings facility
\def\ext@lstlisting{lol}
% show listing number in caption even though \lst@@caption is empty
\def\fnum@lstlisting{\lstlistingname~\thelstlisting}
\makeatother

% TODO: Add new Environments or commands here --> will be externalized later




\begin{document}
\maketitle% this prints the handout title, author, and date

% \begin{abstract}
% \noindent Lernziele:
% \begin{itemize}
% 	\item Kennenlernen der Grundtechniken moderner HTML5 Echtzeitanwendungen
% 	\item Aufbau des WebSocket Protokolls
% 	\item Entwicklung erster WebSocket-basierter Client-Server-Anwendungen
% 	\item Entwicklung einfacher WebSocket-Server mittels JavaScript und dem Vert.x Framework
% \end{itemize}
% \end{abstract}

\begin{lernziele}
\begin{itemize}
	\item Kennenlernen der technischen Beschränkungen von HTTP/1.1
	\item Kennenlernen der Unterschiede zwischen HTTP/1.1 und HTTP/2
	\item Kennenlernen der mit HTTP/2 eingeführten Neuerungen 
\end{itemize}
\end{lernziele}

%\printclassoptions

\setcounter{secnumdepth}{2} % SZA: Added to have numbers in sections

\noindent \rule{1.54\textwidth}{0.4pt}
\tableofcontents
\noindent \rule{1.54\textwidth}{0.4pt}

\section{Preamble}\label{sec:introduction}

% Die Ladezeit einer Webseite ist eines, wenn nicht \emph{das} wichtigste Kriterium einer guten \textbf{Web Usability}. Eine Ladezeit von < 1 Sekunde wird als Heiliger Gral angesehen, nach dem alle News-Sites, Onlineshops, Portale, Blogs und Landingpages suchen. Denn an der Ladezeit hängen sowohl Nutzerzufriedenheit als auch Konversionsraten und deshalb letztlich Traffic und Umsatz einer Seite. Der Einfluss ist so immens, dass beispielsweise Amazon bei einem Anstieg der Ladezeit um 0,1 Sekunden bereits ein Prozent an Umsatz verlieren würde~\citep{heise:2018}.
%
% Das Problem besteht insbesondere auch bei mobilen Webseiten, wie nachfolgender Ausschnitt aus einer 2018 veröffentlichten Studie verdeutlicht~\citep{hobo:2018}:
% \begin{quote}
% 	\emph{``The average time it takes to fully load a mobile landing page is 22 seconds, according to a new analysis. \textbf{Yet 53\% of visits are abandoned if a mobile site takes longer than three seconds to load.} That’s a big problem.''}\begin{flushright}\vspace{-1em}---Google \end{flushright}
% \end{quote}
%
% Google Research hat hierzu weitere Studien durchgeführt (vgl.~\citep{an:2016}) und kommt zu folgenden Ergebnissen:
% \begin{quote}
% 	\emph{``Mobile sites lag behind desktop sites in key engagement metrics such as average time on site, pages per visit, and bounce rate. For retailers, this can be especially costly since 30\% of all online shopping purchases now happen on mobile phones. The average U.S. retail mobile site loaded in 6.9 seconds in July 2016, but, according to the most recent data, \textbf{40\% of consumers will leave a page that takes longer than three seconds to load}. And \textbf{79\% of shoppers who are dissatisfied with site performance} say they're less likely to purchase from the same site again.''}
% \end{quote}
%
% Mit Hilfe eines neuronalen Netzwerks fand man heraus, dass folgende Hauptfaktoren für die Dauer des Seitenaufbaus verantwortlich sind:
% \begin{enumerate}[(i)]
% 	\item Anzahl der Seitenelemente, sowie die
% 	\item Anzahl der enthaltenen Bilder
% \end{enumerate}
% Bei komplexen Seiten mit vielen Elementen dauert das Parsen des Seitencodes sowie der Aufbau des DOMs bedeutend länger. Bei Grafiken empfiehlt es sich, auflösungsoptimierte JPEGs anstatt PNG Formate zu verwenden.
%
% Eine weitere von Google Research durchgeführte Studie (vgl.~\citep{google:2017}), welche die \textbf{Bounce rate}\marginnote{Die \emph{Bounce rate} misst den prozentualen Anteil der Nutzer, die eine Web-Präsenz nach der Startseite wieder verlassen ohne die weiteren Inhalte in Augenschein genommen zu haben (\emph{``[...]without exploring beyond the landing page.''}). } -- also das Verlassen einer Seite ohne deren Inhalt genauer in Augenschein zu nehmen -- untersucht hat, kommt zu folgendem Ergebnis:
%
% \begin{quote}
% 	\emph{``Recently, we trained a deep neural network---a computer system modeled on the human brain and nervous system---with a large set of bounce rate and conversions data. The neural net, which had a 90\% prediction accuracy, found that as page load time goes from one second to seven seconds, the probability of a \textbf{mobile site visitor bouncing increases 113\%}. Similarly, as the number of elements---text, titles, images---on a page goes from 400 to 6,000, the \textbf{probability of conversion drops 95\%}.''}
% \end{quote}


User Interface Design includes both
\begin{enumerate}
	\item Interaction Design
	\item Visual Design
\end{enumerate}

Hence, in order to create intuitive and purposeful interfaces, both areas need to be considered.



%!TEX root = ./ui.tex

\section{Visual Design} % (fold)
\label{sec:visual_design}
This section discusses ...

\subsection{What is Visual Design?} % (fold)
\label{sub:what_is_visual_design}

% subsection what_is_visual_design (end)
What is visual design? Try to think about it a few moments---
\begin{itemize}[]
	\item \emph{...Is it subjective?}
	\item \emph{...Does it require extensive training and talent?}
	\item \emph{...Can only those with creative veins produce good visual designs?}
	\item \emph{...Is it all about graphics?}
\end{itemize}  


\begin{tcolorbox}[
	width=\textwidth,
	% colback={codegray},
	title={\textbf{What is Visual Design?}},
	outer arc=0mm,
	arc=0mm,
	boxrule=1pt,
	% coltitle=white
	]    
Think about it a few moments an try to answer the following questions:
\begin{itemize}[]
	\item \emph{...Is it subjective?}
	\item \emph{...Does it require extensive training and talent?}
	\item \emph{...Can only those with creative veins produce good visual designs?}
	\item \emph{...Is it all about graphics?}
\end{itemize} 
\end{tcolorbox}   


\begin{tcolorbox}[boxsep=0.5em,
                  left=0.5em,
                  right=0pt,
                  top=0pt,
				  outer arc=0mm,
                  arc=0mm,
                  boxrule=0.0pt,leftrule=3pt,
                  colback=exampleback,
				  colframe=examplerule
                  ]%%
What is visual design? Think about it a few moments---
\begin{itemize}[]
	\item \emph{...Is it subjective?}
	\item \emph{...Does it require extensive training and talent?}
	\item \emph{...Can only those with creative veins produce good visual designs?}
	\item \emph{...Is it all about graphics?}
\end{itemize} 
\end{tcolorbox}

People often confuse interface design with graphic design and therefore assume, that UI design is a subjective art that requires extensive training and talent~\citep{mckay:2013}.
This confusion results in two misleading conceptions:

\begin{enumerate}[a)]
	\item The first misconception implies that most of UI design is \emph{graphic design}---which is not true as we will see. 
Although visual design is a crucial element in UI design, most aspects of design efforts should focus on \textbf{interaction design}~\citep{mckay:2013}, ie., how to \textbf{communicate the intended information in meaningful and purposeful ways}.

	\item Another misconception is that visual design is a \emph{subjective art}---this is also not true. 
Although artistic and subjective aspects play a crucial role in contributing to the enjoyable aspect of UX, most of your visual design elements should be selected, created or justified based on \textbf{what they communicate visually}.
\end{enumerate}


As a corollary, many visual design decisions that initially appear subjective, emotional, arbitrary and aesthetic are actually \textbf{objective}, \textbf{rational}, \textbf{coordinated}, and \textbf{principled}~\citep{mckay:2013}. [explain why]

By clarifying the misleading perceptions, we can now define what visual design actually is.
A non-technical but rather comprehensive definition of visual design has been provided by~\citep{usability_gov:2018a}:

\begin{quote}
\emph{``Visual design focuses on the aesthetics of a site and its related materials by strategically implementing images, colors, fonts, and other elements. A successful visual design does not take away from the content on the page or function.  Instead, it enhances it by engaging users and helping to build trust and interest in the brand.''}~\citep{usability_gov:2018a}
\end{quote}

Form may follow function in industrial design, but in designing intuitive user interface, an aesthetic design form follows communication~\citep{mckay:2013}.

\begin{figure}%
	\centering
  \includegraphics[width=1.0\textwidth]{../figures/inconsistent_inputfield.png}
  \caption[][0em]{The length of this numeric text box suggests long input (like a serial or part number), but the spin buttons (the arrows on the right side) suggest that the default value is close to the right value (so unlike a serial or part number). This inconsistency reveals a design problem.\newline Source: \url{https://html5tutorial.info/html5-number.php}}
  \label{fig:inconsistent_inputfield}
\end{figure}

Hence, when evaluating a wireframe, a UI element that does not communicate anything should be removed; if it communicates poorly, it should be redesigned. In cases when a decision has to be made between a design with effective visual communication and one that is aesthetically pleasant, the design that communicates more effectively is the better choice~\citep{mckay:2013}. 


\subsection{Why Visual Design is Important} % (fold)
\label{sub:why_visual_design_is_important}

Note: Good info about Wireframing: https://www.experienceux.co.uk/faqs/what-is-wireframing/

As psychological and anthropological research indicates, visually unappealing products have an impact on \textbf{users subjective perception} about their qualitative characteristics. 
Users will definitely be affected if a product is visually unappealing---regardless of the power and flexibility of the underlying technology. 

Critical in this respect are also defensive statements such as the following:
\begin{quote}
	\emph{``True, our product isn't a beauty, but it surely does the job!''}
	\par
	\emph{``The user interface is a bit old fashioned, but the workflow is very accurate!''}
\end{quote}
Such statements imply that the visual appearance of a product is a superficial detail that users will overlook (or get used to) as long as the required functionality is provided. 
This statement might be true for experts and power users, but it fails the ordinary people.

Why does it fail ordinary people?

The reason why such statements are not true lies in the emotional reactions people feel when using a product.  People are emotional and the react emotionally to a products visual appearance. 
Therefore, a product should look like it fulfills its purpose well.
But if instead the visual appearance of a product is of questionable quality [what does that mean, think about examples], users will naturally assume that the rest of the product has the same level of quality. 
Users assume that attractive products are better designed and more usable---this is known as the \textbf{aesthetic-usability effect}\sidenote{See the resources section for further information.}. As~\citep{mckay:2013} indicates 
\begin{quote}
	\emph{``Do not assume that users will see the beauty that lies beneath---they won't!''}
\end{quote}

As a consequence, visual appearance is essential to our perception of quality~\sidenote{As computer scientists or rational technologists, we are mostly reluctant to accept this idea.}.
We as computer scientists usually want customers to see our software's inner beauty and we want to believe that the quality of the functionality and the internal design---the system architecture, its performance, robustness, reliability, scalability, flexibility etc.---is what matters most. 
They indeed matter, but only for professional or experienced users.
Steve Jobs once said
\begin{quote}
	\emph{``Design is not just what it looks and feels like. Design is how it works.''}
\end{quote}
From a user's point of view, the user experience is the product.
If the user experience is poor, nothing else matters!

\noindent Good visual design is important for a good user experience:
\begin{itemize}
	\item Users need to be spatially oriented and know where to look
	\item Users need to know how to scan a page and how to find the information they are looking for quickly---without being overwhelmed 
	\item The text needs to be readable [dt.~\emph{lesenswert}] and legible [dt.~\emph{leserlich}] (cf.~\emph{typography}) and have a clear visual hierarchy
	\item Users need to understand what icons and graphics mean
	\item They need animations and transitions to keep them oriented, give feedback, and show relationships---without being distracted.
\end{itemize}
A good visual design enables users to get their tasks done effectively and efficiently and without distraction. 


\begin{tcolorbox}[
	width=\textwidth,
	% colback={codegray},
	title={\textbf{Excursus: What is the Aesthetic-Usability Effect?}},
	outer arc=0mm,
	arc=0mm,
	boxrule=1pt,
	% coltitle=white
	]    
Add information about ...
\par \url{https://www.nngroup.com/articles/aesthetic-usability-effect/} 
\par \url{https://medium.com/coffee-and-junk/design-psychology-aesthetic-usability-effect-494ed0f22571}
\end{tcolorbox}



%!TEX root = ./ui.tex

\section{Design Thinking } % (fold)
\label{sec:visual_design}
This section answers the two questions: 
\begin{enumerate}[a)]
	\item \emph{What is design thinking, and}
	\item \emph{Why is it important}
\end{enumerate}


\subsection{What is Design Thinking?} % (fold)
\label{sub:what_is_design_thinking}

[Definition] 


\begin{tcolorbox}[
	width=\textwidth,
	% colback={codegray},
	title={\textbf{A Design Thinking Problem Solving Example}},
	outer arc=0mm,
	arc=0mm,
	boxrule=1pt,
	% coltitle=white
	]    
Thinking outside of the box can provide an innovative solution to a sticky problem. However, thinking outside of the box can be a real challenge as we naturally develop patterns of thinking that are modelled on the repetitive activities and commonly accessed knowledge we surround ourselves with. It takes something to break away from a situation where we’re too closely involved to be able to find better possibilities. 
\par \vspace{0.5em}
To illustrate how a fresh way of thinking can create unexpectedly good solutions, let’s look at a famous story. Some years ago, an incident occurred where a truck driver had tried to pass under a low bridge. Alas, he failed, and the truck became firmly lodged under the bridge. The driver was unable to continue driving through or reverse out.
\par \vspace{0.5em}
The story goes that as the truck became stuck, it caused massive traffic problems, which resulted in emergency personnel, engineers, firefighters, and truck drivers gathering to negotiate various solutions so as to dislodge the truck.
\par \vspace{0.5em}
Emergency workers were debating whether to dismantle parts of the truck or chip away at parts of the bridge. Each spoke of a solution which fitted within his or her respective level of expertise. In the heat of the emergency, all parties carried on with their ways of viewing the problem, including the truck driver, whose initial dismay over a scraped roof had turned into a deeper concern.
\par \vspace{0.5em}
A boy walking by and witnessing the intense debate looked at the truck, at the bridge, then looked at the road and said nonchalantly, \emph{``Why not just let the air out of the tires?''} to the absolute amazement of all the specialists and experts trying to unpick the problem.
\par \vspace{0.5em}
When the solution was tested, the truck was able to drive free with ease, having suffered only the damage caused by its initial attempt to pass underneath the bridge. Whether or not the story actually happened in real life, it symbolizes the struggles we face where oftentimes the most obvious solutions are the ones hardest to come by because of the self-imposed constraints we work within.
\end{tcolorbox}   

Challenging our assumptions and everyday knowledge is often difficult for us humans, as we rely on building patterns of thinking in order not to have to learn everything from scratch every time. We rely on doing everyday processes more or less unconsciously---for example, when we get up in the morning, eat, walk, and read---but also when we assess challenges at work and in our private lives. Especially experts and specialists rely on their solid thought patterns, patterns that serve them well in their respective fields, not to mention the people to whom they deliver their skills. Even so, it can be very challenging and difficult for experts to start questioning their knowledge. Pride aside, it can prove more than a little disconcerting to think that many years of education and practical experience can hinder rather than help in dealing with a problem.


Design thinking is firmly based on generating a \textbf{holistic and emphatic understanding} of the problems that people face, and that it involves ambiguous or inherently subjective concepts such as emotions, needs, motivations, and drivers of behaviors.
This contrasts with a solely scientific approach, where there’s more of a distance in the process of understanding and testing the user’s needs and emotions---e.g., via quantitative research. Tim Brown sums up that design thinking is a third way: design thinking is essentially a problem-solving approach, crystallized in the field of design, which combines a holistic user-centered perspective with rational and analytical research with the goal of creating innovative solutions.

Design thinking is an iterative and non-linear process. This simply means that the design team continuously use their results to review, question, and improve their initial assumptions, understandings and results. Results from the final stage of the initial work process inform our understanding of the problem, help us determine the parameters of the problem, enable us to redefine the problem, and, perhaps most importantly, provide us with new insights so we can see any alternative solutions that might not have been available with our previous level of understanding.

\section{Summary} % (fold)
\label{sec:summary}
Design thinking is essentially a problem-solving approach specific to design, which involves assessing known aspects of a problem and identifying the more ambiguous or peripheral factors that contribute to the conditions of a problem. This contrasts with a more scientific approach where the concrete and known aspects are tested in order to arrive at a solution. Design thinking is an iterative process in which knowledge is constantly being questioned and acquired so it can help us redefine a problem in an attempt to identify alternative strategies and solutions that might not be instantly apparent with our initial level of understanding. Design thinking is often referred to as ‘outside the box thinking’, as designers are attempting to develop new ways of thinking that do not abide by the dominant or more common problem-solving methods – just like artists do. At the heart of design thinking is the intention to improve products by analyzing how users interact with products and investigating the conditions in which they operate. Design thinking offers us a means of digging that bit deeper to uncover ways of improving user experiences.




\newpage
\bibliography{refs}
\bibliographystyle{plainnat} % SZA: was plainnat




\end{document}