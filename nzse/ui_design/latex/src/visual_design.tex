%!TEX root = ./ui.tex

\section{Visual Design} % (fold)
\label{sec:visual_design}
This section discusses ...

\subsection{What is Visual Design?} % (fold)
\label{sub:what_is_visual_design}

% subsection what_is_visual_design (end)
What is visual design? Try to think about it a few moments---
\begin{itemize}[]
	\item \emph{...Is it subjective?}
	\item \emph{...Does it require extensive training and talent?}
	\item \emph{...Can only those with creative veins produce good visual designs?}
	\item \emph{...Is it all about graphics?}
\end{itemize}  


\begin{tcolorbox}[
	width=\textwidth,
	% colback={codegray},
	title={\textbf{What is Visual Design?}},
	outer arc=0mm,
	arc=0mm,
	boxrule=1pt,
	% coltitle=white
	]    
Think about it a few moments an try to answer the following questions:
\begin{itemize}[]
	\item \emph{...Is it subjective?}
	\item \emph{...Does it require extensive training and talent?}
	\item \emph{...Can only those with creative veins produce good visual designs?}
	\item \emph{...Is it all about graphics?}
\end{itemize} 
\end{tcolorbox}   


\begin{tcolorbox}[boxsep=0.5em,
                  left=0.5em,
                  right=0pt,
                  top=0pt,
				  outer arc=0mm,
                  arc=0mm,
                  boxrule=0.0pt,leftrule=3pt,
                  colback=exampleback,
				  colframe=examplerule
                  ]%%
What is visual design? Think about it a few moments---
\begin{itemize}[]
	\item \emph{...Is it subjective?}
	\item \emph{...Does it require extensive training and talent?}
	\item \emph{...Can only those with creative veins produce good visual designs?}
	\item \emph{...Is it all about graphics?}
\end{itemize} 
\end{tcolorbox}

People often confuse interface design with graphic design and therefore assume, that UI design is a subjective art that requires extensive training and talent~\citep{mckay:2013}.
This confusion results in two misleading conceptions:

\begin{enumerate}[a)]
	\item The first misconception implies that most of UI design is \emph{graphic design}---which is not true as we will see. 
Although visual design is a crucial element in UI design, most aspects of design efforts should focus on \textbf{interaction design}~\citep{mckay:2013}, ie., how to \textbf{communicate the intended information in meaningful and purposeful ways}.

	\item Another misconception is that visual design is a \emph{subjective art}---this is also not true. 
Although artistic and subjective aspects play a crucial role in contributing to the enjoyable aspect of UX, most of your visual design elements should be selected, created or justified based on \textbf{what they communicate visually}.
\end{enumerate}


As a corollary, many visual design decisions that initially appear subjective, emotional, arbitrary and aesthetic are actually \textbf{objective}, \textbf{rational}, \textbf{coordinated}, and \textbf{principled}~\citep{mckay:2013}. [explain why]

By clarifying the misleading perceptions, we can now define what visual design actually is.
A non-technical but rather comprehensive definition of visual design has been provided by~\citep{usability_gov:2018a}:

\begin{quote}
\emph{``Visual design focuses on the aesthetics of a site and its related materials by strategically implementing images, colors, fonts, and other elements. A successful visual design does not take away from the content on the page or function.  Instead, it enhances it by engaging users and helping to build trust and interest in the brand.''}~\citep{usability_gov:2018a}
\end{quote}

Form may follow function in industrial design, but in designing intuitive user interface, an aesthetic design form follows communication~\citep{mckay:2013}.

\begin{figure}%
	\centering
  \includegraphics[width=1.0\textwidth]{../figures/inconsistent_inputfield.png}
  \caption[][0em]{The length of this numeric text box suggests long input (like a serial or part number), but the spin buttons (the arrows on the right side) suggest that the default value is close to the right value (so unlike a serial or part number). This inconsistency reveals a design problem.\newline Source: \url{https://html5tutorial.info/html5-number.php}}
  \label{fig:inconsistent_inputfield}
\end{figure}

Hence, when evaluating a wireframe, a UI element that does not communicate anything should be removed; if it communicates poorly, it should be redesigned. In cases when a decision has to be made between a design with effective visual communication and one that is aesthetically pleasant, the design that communicates more effectively is the better choice~\citep{mckay:2013}. 


\subsection{Why Visual Design is Important} % (fold)
\label{sub:why_visual_design_is_important}

Note: Good info about Wireframing: https://www.experienceux.co.uk/faqs/what-is-wireframing/

As psychological and anthropological research indicates, visually unappealing products have an impact on \textbf{users subjective perception} about their qualitative characteristics. 
Users will definitely be affected if a product is visually unappealing---regardless of the power and flexibility of the underlying technology. 

Critical in this respect are also defensive statements such as the following:
\begin{quote}
	\emph{``True, our product isn't a beauty, but it surely does the job!''}
	\par
	\emph{``The user interface is a bit old fashioned, but the workflow is very accurate!''}
	\par
	\emph{``Users will focus on the functionality not the UI!''}
	\par
	\emph{``If the system works, the UI is of subordinate importance!''}
\end{quote}
Such statements imply that the visual appearance of a product is a superficial detail that users will overlook (or get used to) as long as the required functionality is provided. 
This statement might be true for experts and power users, but it fails the ordinary people.

Why does it fail ordinary people?

The reason why such statements are not true lies in the emotional reactions people feel when using a product. \textbf{People are emotional and they react emotionally to a products visual appearance}. 
Therefore, a product should look like it fulfills its purpose well.
But if instead the visual appearance of a product is of questionable quality\sidenote{Q: What does that mean, think about examples.}, users will naturally assume that the rest of the product employs the same level of quality. 
Users assume that attractive products are better designed and more usable---this is known as the \textbf{aesthetic-usability effect}\sidenote{See the resources section for further information.}. As~\citep{mckay:2013} indicates 
\begin{quote}
	\emph{``Do not assume that users will see the beauty that lies beneath---they won't!''}
\end{quote}

As a consequence, visual appearance is essential to our perception of quality~\sidenote{As computer scientists or rational technologists, we are mostly reluctant to accept this idea.}.
We as computer scientists usually want customers to see our software's inner beauty and we want to believe that the quality of the functionality and the internal design---the system architecture, its performance, robustness, reliability, scalability, flexibility etc.---is what matters most. 
They indeed matter, but only for professional or experienced users.
Steve Jobs once said
\begin{quote}
	\emph{``Design is not just what it looks and feels like. Design is how it works.''}
\end{quote}
From a user's point of view, the user experience is the product.
If the user experience is poor, nothing else matters!

\noindent Good visual design is important for a good user experience:
\begin{itemize}
	\item Users need to be spatially oriented and know where to look
	\item Users need to know how to scan a page and how to find the information they are looking for quickly---without being overwhelmed 
	\item The text needs to be readable [dt.~\emph{lesenswert}] and legible [dt.~\emph{leserlich}] (cf.~\emph{typography}) and have a clear visual hierarchy
	\item Users need to understand what icons and graphics mean
	\item They need animations and transitions to keep them oriented, give feedback, and show relationships---without being distracted.
\end{itemize}
A good visual design enables users to get their tasks done effectively and efficiently and without distraction. 


\begin{tcolorbox}[
	width=\textwidth,
	% colback={codegray},
	title={\textbf{Excursus: What is the Aesthetic-Usability Effect?}},
	outer arc=0mm,
	arc=0mm,
	boxrule=1pt,
	% coltitle=white
	]    
Add information about ...
\par \url{https://www.nngroup.com/articles/aesthetic-usability-effect/} 
\par \url{https://medium.com/coffee-and-junk/design-psychology-aesthetic-usability-effect-494ed0f22571}
\end{tcolorbox}

% \section{Design Guidelines} % SZA: Was previous headline
\label{sec:design_guidelines}

\section{Layout Design Principles} % (fold)
\label{sub:layout_design_principles}
Layout is one of the hardest parts in UI design---especially for non-designers.
Here are some guidelines that help in designing layouts, that appear simple and orderly (ie., clearly structured), easy to scan, efficient, and aesthetically leasing.

Several layout patterns are often recommended to take advantage of how people scan or read through a design. 3 of the more common are the Gutenberg diagram, the z-pattern layout, and the f-pattern layout. Each offers advice for where to place important information.



\section{Reading Patterns} % (fold)
\label{sub:reading_patterns}
(Source: \url{https://www.nngroup.com/articles/website-reading/})

Apart from\marginnote{Reading behaviour on the Web is paradoxial} direct transactions such as online banking or grading information, user behavior in relation to Web content in particular is paradoxical:
\begin{enumerate}
	\item Users go to websites for \textbf{information}, but
	\item Users \textbf{scarcely read anything} during an average website visit.
\end{enumerate}

This second point has been well supported by vast amounts of research over the last years:
\begin{enumerate}
	\item In\marginnote{Users do not read on the Web} 1997, the world's first study of how users read web content summarized the findings in two words: \textbf{they don't}. Instead of carefully reading information, users typically \textbf{scan} it.
	\item In\marginnote{Users scan using an F-pattern} 2006, eyetracking research found that users frequently scan website prose in an \textbf{F-shaped pattern}, focusing on words at the top or left side of the page, while barely glancing at words that appeared elsewhere.
	\item 2008\marginnote{Users only read 28\% of page content} research quantified this finding: given the duration of an average page view, users have time to read at most \textbf{28\% of the words on the page}.
\end{enumerate}



% For reading, we generally distinguish between different forms of reading and scanning patterns (different patterns)
% immersive reading
% scanning
% F-shaped reading
% Z-shaped reading
% hybrid forms
%
% We will discuss ... and ... in detail and explain ... through examples.

There are two modes of reading, ie., comprehending the contents of a screen or Web page:
\begin{enumerate}
	\item \textbf{Immersive reading} \par The\marginnote{The goal of immersive reading is comprehension} goal of immersive reading is \textbf{comprehension}, ie., understanding the semantics and consequences of a text thoroughly.
	Users read immersively in a left-to-right, top-to-bottom order (in Western cultures).
	During immersive reading, users read most of the words and most of the content, but they may skip over content that does not appear relevant or that requires too much effort to read.
	
	\item \textbf{Scanning} \par The\marginnote{Scanning is used for finding desired information quickly and filtering irrelevant or unrelated content.} goal of scanning is finding desired things quickly with as little effort as possible.
	Users often scan a page using an \textbf{arching pattern}, starting in the upper-left corner and ending in the lower right.
	This scanning path is formally known as the \textbf{Gutenberg Diagram} (see Figure~\ref{fig:gutenberg-diagram}).
	
\end{enumerate}

\begin{figure}%
  \includegraphics[width=1.0\textwidth]{../figures/gutenberg-diagram.jpg}
  \caption[][0em]{The Gutenberg Diagram is a visualization of how Latin-alphabet readers---those of us who read from left-to-right, top-to-bottom---process information presented on a webpage. (Source: \url{https://www.clicksandclients.com/blog/internet-marketing/gutenberg-diagram-homepage-sense/})}
  \label{fig:gutenberg-diagram}
\end{figure}

Users follow these patterns generally, but the patterns change when there is content that attracts or repulses attention (---> see \emph{banner blindness}).

Well-designed layouts need to accommodate both modes well. 

\begin{tcolorbox}[
	width=\textwidth,
	% colback={codegray},
	title={\textbf{Question}},
	outer arc=0mm,
	arc=0mm,
	boxrule=1pt,
	% coltitle=white
	]    
How can a design enable or support both reading modes?
\end{tcolorbox} 


\subsection{Scanning on Mobile Devices} % (fold)
\label{ssub:scanning_on_mobile_devices}

Small screens found on smart phones have a different scanning pattern. Users scan small screens starting in the upper left corner and going straight down. They can take in the width of the screen without scanning across.

\subsection{Banner Blindness} % (fold)
\label{sub:banner_blindness}

% subsection banner_blindness (end)
When reading the contents of a page, users skip over any UI elements that repulse their attention, e.g., anything that looks like an advertisement. This phenomenon is known as \textbf{banner blindness}, in which users often skip over banners because they assume they are adds (this is ironic because banners are intended to attract attention).







\section{Scanning Patterns} % (fold)
\label{sec:scanning_patterns}
Good source for discussion about pattern: \url{https://vanseodesign.com/web-design/3-design-layouts/}

\subsection{F-Shaped Pattern}\sidenote{Source: \url{https://uxplanet.org/f-shaped-pattern-for-reading-content-80af79cd3394}} % (fold)
\label{sub:f_shaped_pattern}

An F-shaped pattern is used on text- or content-rich pages (---> see examples for NYT, CNBC etc).
% The F-Pattern describes the most common user eye-scanning patterns when it comes to blocks of content. F for fast. That’s how users read your content. In a few seconds, their eyes move at amazing speeds across your website's page.

\begin{figure}%
	\centering
  \includegraphics[width=1\textwidth]{../figures/f-pattern_screens.jpeg}
  \caption[][0em]{The NNGroup demonstrates how eye-tracking studies revealed that users (in left-to-right reading cultures) typically scan heavy blocks of content in a pattern that looks like the letter F or E. The areas where users looked the most are colored red; the yellow areas indicate fewer views, followed by the least-viewed blue areas. Gray areas did not attract any fixations. \par (Source: \url{https://uxplanet.org/f-shaped-pattern-for-reading-content-80af79cd3394})}
  \label{fig:f-layout}
\end{figure}

The pattern was popularized by NNGroup eye tracking study which recorded more than 200 users looked at thousands of web pages and found that users’ main reading behavior was fairly consistent across many different sites and tasks. This reading pattern applied by users looked somewhat like the ``F''-letter and has the following three characteristics (see Figure~\ref{fig:f-layout}):

\begin{itemize}
	\item Users first read in a horizontal movement, usually across the upper part of the content area. This initial element forms the F’s top bar.
	
	\item Next, they scan a vertical line down the left side of the screen, looking for points of interest in the paragraph’s initial sentences. When they found something interesting they read across in a second horizontal movement that typically covers a shorter area than the previous movement. This additional element forms the F’s lower bar.
	
	\item Finally, users scan the content’s left side in a vertical movement.
\end{itemize}

\begin{figure}%
  \includegraphics[width=1\textwidth]{../figures/f-pattern.png}
  \caption[][0em]{Our eyes are trained to start at the top-left corner, scan horizontally, then drop down to the next line and do the same until we find something of interest.}
  \label{fig:f-layout}
\end{figure}

Obviously,\marginnote{Users omit the `F'-pattern if something else attracts their interest.} users’ scan patterns are not always comprised of exactly three parts. When the reader finds something they like, they begin reading normally, forming horizontal lines.

\paragraph{Why should it be used?}
F-shaped pattern will help you create a design with good visual hierarchy, a design that people can scan easily. F-shaped layout feels comfortable for the most western readers, because they have been reading top to bottom, left to right for their entire lives.

\paragraph{When should it be used?}
The F-pattern is the go-to layout for text-heavy websites like blogs and news sites. If there is a lot of content---especially text---users will respond better with the layout that designed according to the natural scanning format.


The F-Layout gives the designer more control over what gets seen.
It is therefore important to prioritize the content from most to least important before arranging elements. Once it is clear what you want your users to see, you can simply place them in the pattern’s ‘hot spots’ for the right interactions.

Bear in mind, that the first two paragraphs are the most important. 
Therefore, the most important content should be placed near to the top of the page as possible in an attempt to communicate the site’s (or page) purpose quickly. The user will usually read horizontally across the header, so here’s a good place for a navigation bar.


\subsection{Z-Shaped Pattern} % (fold)
\label{sub:z_shaped_pattern}

A z-pattern design traces the route the human eye travels when they scan the page— left to right, top to bottom:
\begin{enumerate}
	\item First, people scan from the top left to the top right, forming a horizontal line
	\item Next, down and to the left side of the page, creating a diagonal line
	\item Last, back across to the right again, forming a second horizontal line
\end{enumerate}
When viewers’ eyes move in this pattern, it forms an imaginary ``Z'' shape.
This pattern works because most western readers will scan a page the same way that they would scan a book---top to bottom, left to right.

Z-Pattern\marginnote{Where to use the Z-Pattern.} scanning occurs on pages that aren’t centered on the text (for text-heavy pages such as articles or search results, it’s better to use F-Pattern). This makes z-pattern good solution for simple designs with a minimal copy and a few key elements that need to be seen. Minimalist pages or landing pages centering mostly around one or two main elements can implement the Z-pattern to encourage users to follow this natural method.

\begin{figure}%
	\centering
  \includegraphics[width=1.55\textwidth]{../figures/z-shape_facebook.png}
  \caption[][24em]{Z-layout truly shines in design projects where simplicity and a call-to-action are the most important principle. Facebook landing page is an example of Z-Layout.\newline Source: \url{https://uxplanet.org/z-shaped-pattern-for-reading-web-content-ce1135f92f1c}}
  \label{fig:z-shape_facebook}
\end{figure}

The premise of the Z-layout is actually pretty simple: impose the letter Z on the page. Ideally, you want people to see your most important information first and your next most important information second. Thus, important elements should be placed along the scanning path and visitors should be presented with the right information at the right time.

It’s essential to create a flow

Flow is about leading the eye from one part of a page to another in the direction you want it to move. You create flow through a combination of visual weight and visual direction. Here are a few best practices to keep in mind when creating a flow:

\begin{enumerate}[(a)]
	\item \textbf{Point \#1 --- Logo} \par Point \#1 is a starting point of viewer’s journey. It’s a prime location for your logo.
	\item \textbf{Point \#2 --- Search or Login Elements} \par Place the items that you want the reader to see first along the top of the Z. The eye will naturally follow the path of the Z, so the goal is to place your secondary ``call to action'' at the end. Put more visual weight into Point \#2 element---make the button (or another key element) bright and colorful to get users attention and guide users along the Z-pattern.
	\item \textbf{Center area of the page --- Hero image or main message} \par The trick to this area is fill it with content that interests the user, while still urging their sight downward to the next line. For example, you can place a hero image in the center of the page to separate the top and bottom sections and guide the eyes along the Z path.
	\item \textbf{Point \#3 --- Useful information} \par The purpose of Point \#3 is to guide the users to the final call to action at Point \#4. For example, if your page promotes some product that you want to sell, you want potential customers to see the copy that will convince them to buy before they see the ``Buy Now'' button. Thus, you can use Point \#3 to provide benefits or other helpful information for them.
	\item \textbf{Point \#4 --- Action UI Elements} \par Point \#4 is the finish line, the row between it and Point 3 should contain content that pushes the user’s sight to the corner. Point \#4 itself is an ideal place for your primary Call to Action.
\end{enumerate}


\begin{figure}%
	\centering
  \includegraphics[width=1\textwidth]{../figures/z-shape.png}
  \caption[][0em]{The Z-layout}
  \label{fig:z-layout}
\end{figure}
	

Below you can see two great example of Z-layout from Basecamp and Evernote.
% \newpage % looks good!	
\begin{figure}%
	\centering
  \includegraphics[width=1.5\textwidth]{../figures/z-shape_basecamp.png}
  \caption[][27em]{Z-layout used on the basecamp.com landing page. (Source: \url{https://basecamp.com/})}
  \label{fig:z-shape_facebook}
\end{figure}

\begin{figure}%
	\centering
  \includegraphics[width=1.5\textwidth]{../figures/z-shape_evernote.png}
  \caption[][21em]{Z-layout used on the evernote.com landing page.\newline Source: \url{https://evernote.com/}}
  \label{fig:z-shape_facebook}
\end{figure}

\subsection{Zig-Zag-Shaped Pattern} % (fold)
\label{sub:zig_zag_shaped_pattern}

% section zig_zag_shaped_pattern (end)Zig-Zag Pattern}
The interesting and useful thing to know about the Z-pattern is that we can extend this pattern a little by seeing it more as a series of z-movements instead of one big z-movement.

As you can see below, this is exactly what Dropbox does by guiding users through a few key product features and finishing their repeated Z-pattern with “Sign Up For Free” call-to-action button. In this layout “Learn More” buttons play a role of secondary call-to-action buttons that help readers go to the next relevant page without needing to read a full copy.

\begin{figure}%
	% \centering
  \includegraphics[width=1.0\textwidth]{../figures/zig-zag_dropbox.png}
  \caption[][0em]{Zig-Zag-layout used on the dropbox.com landing page.\newline Source: \url{https://basecamp.com/}}
  \label{fig:z-shape_facebook}
\end{figure}

The new evernote.com landing page also adapted to the improved zig-zag-layout.

\begin{figure}%
	% \centering
  \includegraphics[width=1.0\textwidth]{../figures/zig-zag-shape_evernote.png}
  \caption[][0em]{Zig-Zag-layout used on the new evernote.com landing page.\newline Source: \url{https://basecamp.com/}}
  \label{fig:z-shape_facebook}
\end{figure}










\section{Content Usability} % (fold)
\label{sec:content_usability}

% section content_usability (end)

A well-designed page help users save time by focusing their scanning eyes on information that they actually want to read (--> user research).

It is important to note that when web content helps users focus on sections of interest, users switch from scanning to actually reading the copy.

Therefore it is important to place important and significant information first
a) First 2 Words in Headlines\sidenote{Source: \url{https://www.nngroup.com/articles/first-2-words-a-signal-for-scanning/}}
b) First 2 Paragraphs in texts should contain the most important and relevant information.

Users pay the \textbf{most attention to information at the top} of the page.

Experiments conducted by the NNGroup revealed that users assess the importance and relevance of text based on its relative distance to the top headline. 

The following table shows the percentage of users who looked at a paragraph relative to its sequential location in the body text:

Position from the start of the text	Users who looked at the paragraph
1	81\%
2	71\%
3	63\%
4	32\%



\subsection{Facts First}\marginnote[-1em]{or:\par \emph{``How to create a good Content Usability''}} % (fold)
\label{sub:fact_hunting}


% subsection fact_hunting (end)
Always bear in mind that users hunt for facts online, so \textbf{factually rich content} will attract readers and keep their attention\sidenote{Source: \url{https://www.nngroup.com/articles/write-interesting-facts/}}.

Central problem:

Given that users spend almost no time visiting the average web page, how do you get people to actually read your website pages?

Users often leave Web pages in 10–20 seconds, but pages with a clear value proposition can hold people's attention for much longer. To gain several minutes of user attention, you must clearly communicate your value proposition within 10 seconds.\sidenote{\url{https://www.nngroup.com/articles/how-long-do-users-stay-on-web-pages/} ---very interesting read}

[add info about negative aging effect of website attractiveness]

\subsection{Put the Introduction right} % (fold)
\label{sub:put_the_introduction_right}

Introductory text on Web pages is usually too long, so users skip it. But short intros can increase usability by explaining the remaining content's purpose\sidenote{Source: \url{https://www.nngroup.com/articles/blah-blah-text-keep-cut-or-kill/}}.

The introductory paragraph(s) found at the top of many Web pages often contain general and unspecific information: a block of words that users typically skip when they arrive at a page. Instead, their eyes go directly to more actionable content, such as product features, bulleted lists, or hypertext links.

\begin{quote}
\emph{``People read very little on Web pages. Don't waste word count on generic, feel-good material. It's not going to make customers feel good anyway. They care only about getting their problems solved as quickly as possible so they can leave your site.''}~\sidenote{\url{https://www.nngroup.com/articles/blah-blah-text-keep-cut-or-kill/}}
\flushright{---Jakob Nielsen, Blah-Blah Text: Keep, Cut, or Kill?}
\end{quote}

Why not just eliminate any introductory text?

Intro text has a valid role in that it helps set the context for content and thus answer the question: What's the page about?

A brief introduction can help users better understand the rest of the page. Even if they skip it initially, they might return later if it doesn't look intimidatingly long and dense. If you keep it \textbf{short}, a bit of blah might actually work. So, prune your initial draft of marketese and focus on answering two questions:

\begin{itemize}
	\item What? (What will users find on this page --- i.e., what's its function?)
	\item Why? (Why should they care --- i.e., what's in it for them?)
\end{itemize}

For improving content usability, we discussed two approaches
a) factual texts
b) concise intros

\begin{figure}%
	% \centering
  \includegraphics[width=1.50\textwidth]{../figures/good_intro_nngroup.png}
  \caption[][24em]{Articles published on Nielsen Norman Group provide a clear and concise introductory text in form of a one sentence summary (indicated by the indented red colored text). Also note the clear ---> \emph{visual hierarchy}.}
  \label{fig:content_usability_nng}
\end{figure}

When we look at this example, what do we see
a) big title
b) date
c) topics -- factual knowledge
d) summary statement / intro. important: mind the context-information (``Summary: ...'') --- it clearly indicates what this text is, what it is about and what users can expect from it.

\begin{figure}%
  \includegraphics[width=1.54\textwidth]{../figures/content_usability_bamf.png}
  \caption[][57em]{Content usability of an article about the ``Digitalisierungsagenda 2020'' published by the German Federal Office for Migration and Refugees (BAMF). (Accessed September 27th, 2019.) }
  \label{fig:content_usability_bamf}
\end{figure}

\begin{figure}%
  \includegraphics[width=1.54\textwidth]{../figures/content_usability_kfw.png}
  \caption[][46em]{Content usability of an article about the ``Digitalisierungsagenda 2020'' published by the German Federal Office for Migration and Refugees (BAMF). (Accessed September 27th, 2019.) }
  \label{fig:content_usability_kfw}
\end{figure}




\subsection{Progressive Disclosure} % (fold)
\label{sub:progressive_disclosure}
% ger: fortschreitende Offenlegung

[Very good article about progressive disclosure: \url{https://vanseodesign.com/web-design/progressive-discolosure/}]

Central question:
\begin{itemize}
	\item How to present information online to the appropriate extent?
	\item What is the underlying conversation model?
	\item Would the conversation more like a monologue where everything will be explained in detail\sidenote{like in most university lectures}?
	\item Or would the conversation more like a brief summary of the essentials followed by Q\&A?
\end{itemize}
 

To correctly answer these questions, think about how you would communicate a specific kind of information in person in the real world. Think about your target audience and how you would communicate to them in appropriate ways.
If a summary of the essential facts followed by a progressive disclosure of optional information is the most appropriate communication model, then progressive disclosure is a good communication model to use.

Progressive disclosure approach works well with the \textbf{inverted pyramid presentation style} invented by journalism, where the most important information is placed first and details follow according to their importance. Using this style, users can stop reading at any time once they have the information they need.

[TODO: add image of inverted pyramid]


[TODO: add image of trip advisor or amazon stars]


Progressive disclosure is an interaction design technique often used in human computer interaction to help maintain the focus of a user's attention by reducing clutter, confusion, and cognitive workload. This improves usability by presenting only the minimum data required for the task at hand.

The idea is to help prevent information overload and keep %designs cleaner by reducing clutter and noise. The goal is to keep
 your audience from becoming frustrated or disoriented by giving them what they need and want and nothing more.

From his 2006 Alertbox article on progressive disclosure, Jacob Nielson points out the dilemma faced by designers.

\begin{enumerate}[(a)]
	\item Users want power, \textbf{features}, and \textbf{enough options} to handle all of their special needs.
	\item Users want \textbf{simplicity}; they don’t have time learn a profusion of features in enough depth to select the few that are optimal for their needs.
\end{enumerate}

How does one provide all the features users want while still keeping interfaces clean and simple? In the same post Jacob also offers the solution (see \url{http://www.useit.com/alertbox/progressive-disclosure.html}).

\begin{enumerate}[(a)]
	\item Initially, show users only a few of the most important options.
	\item Offer a larger set of specialized options upon request. Disclose these secondary features only if a user asks for them, meaning that most users can proceed with their tasks without worrying about this added complexity.
\end{enumerate}

This is progressive disclosure. You want to separate information into multiple layers and by default only present those layers that are necessary or relevant to the task at hand. Along with those necessary layers of information you want to provide simple mechanisms to let people have more information on request.

There are two related ideas in the previous paragraph. One is to provide necessary and relevant default information and the other is to provide more on request.

The first idea tells us it’s ok to present different information on different pages of a web site and the second tells us we should provide a way for our visitors to get more information if they want. This extra information might also change from page to page, though all the extra information should be readily available at any and all times.

Progressive disclosure enables you to hide advanced information from new users while still making that information available to the more advanced users who want it. Information that isn’t currently wanted is essentially noise. Reducing noise and increasing signal is something we should strive for in design and progressive disclosure gives us a mechanism for doing so.

Presenting limited sets of information increases learning efficiency. When information is gradually presented as needed or requested it’s processed better and is perceived as more relevant. Errors and the time and frustration of recovering from those errors is consequently reduced. Less frustration means a better user experience.

\begin{figure}%
  \includegraphics[width=1.0\textwidth]{../figures/firewall_start.png}
  \caption[][0em]{Opening screen of the firewall option in the system preferences of Mac OS 10.11. The elementary information und functionality is clearly visible.}
  \label{fig:firewall_start}
\end{figure}

\begin{figure}%
  \includegraphics[width=1.0\textwidth]{../figures/firewall_details.png}
  \caption[][0em]{The firewall option screen Mac OS 10.11. Details and expert options are displayed on a secondary screen in order to minimize clutter and information overload.}
  \label{fig:firewall_details}
\end{figure}



\subsection{Signal-to-Noise Ratio} % (fold)
\label{sub:signal_to_noise_ratio}
[Source: \url{https://vanseodesign.com/web-design/signal-to-noise-ratio/}]
% Include pollin and reichelt web shop images

% compare it to apple 

The concept of signal-to-noise ratio began as an abstract electrical engineering equation but has since evolved into a useful metaphor for many kinds of communication.

All communication is a chain of creation, transmission, and reception of information. At each step along the way, the useful information—the signal—is degraded by extraneous or irrelevant information: the noise. Good communication thus maximizes what’s important while minimizing the things that distract from the message.

There are a series of steps in any communication. It begins with the creation of a message, followed by the transmission of that message, and ending with the reception of that message by another party. At each of these 3 stages the signal itself degrades some and noise is added.

In analog and digital communication signal-to-noise ratio (S/N or SNR) is a measure of the signal strength that is received relative to the background noise that is also received. If you’ve ever tried to fine tune the reception of an AM/FM radio so you could better hear a station amidst the static you were doing what you could to increase the signal-to-noise ratio.

For our purposes as designers the signal is the information you want to communicate to your audience, the message you are trying to convey. Noise is pretty much everything else. It’s all the extraneous information that doesn’t serve to communicate your message.

The higher the signal-to-noise ratio of your design, the more clearly your message is communicated to those viewing your design. Your goal with every design should be to strive for the maximum S/N possible.

Case Study: Some eCommerce sites

[add images of pollin, reichelt etc. and compare it to apple] 


\subsection{Further Recommendations} % (fold)
\label{sub:general_recommendations}

Some recommendations
Whenever possible, put \textbf{crucial text on interactive controls} instead of using static text labels. Users are less likely to scan static text, so they might miss that crucial information.

\textbf{Reconsider banners} and \textbf{large blocks of} mostly \textbf{unformatted text}. These formats are often used for important information, but users are likely to not scan them at all.

Place primary UI elements required to perform the task prominently along the scan axis between the starting and ending points.

Consider using bold labels for editable fields because users are likely to scan for specific labels. By contrast, do not use bold labels for reports because users are more likely to scan for specific data, not the labels.

[TODO: Add image p.156]

[TODO: Add infos from summary and other sections there ]





\section{Alignment} % (fold)
\label{sec:alignment}

\marginnote{An excellent online article about layout grids can be retrieved from: \url{https://www.smashingmagazine.com/2017/12/building-better-ui-designs-layout-grids/} (accessed: Sept. 30th, 2018) }

When designing the visual appearance of a software, use an invisible alignment system to give the UI elements a coordinated and orderly appearance. 
Layouts can be improved by reducing the number of vertical alignment grids. 
The reuse of alignment templates across a software ensures a consistent appearance.

[TODO: Add illustration about alignment grids; consider gutenberg diagramm ]

Label alignment defines the topological relationships between controls and their labels. The most common styles are 
\begin{enumerate}[(a)]
	\item top-aligned labels,
	\item left-aligned labels,
	\item right-aligned labels, and
	\item left-aligned labels with ragged controls.
\end{enumerate}
For larger screens, there is no single best alignment approach; each has benefits that depend on how the controls and their labels are used. 


\begin{figure}%
  \includegraphics[width=\textwidth]{../figures/vertical_align.png}
  \caption[][0em]{Top-aligned labels work best when the controls are interactive (such as forms), when users are likely to scan both the labels and the controls vertically, and when ease of localization is important. A downside is that top-aligned labels take a lot of vertical space, so they do not work well with many fields. \newline Picture source: \url{https://www.lukew.com/ff/entry.asp?1502}}
  \label{fig:vertical_align}
\end{figure}

\begin{figure}%
  \includegraphics[width=\textwidth]{../figures/left_align.png}
  \caption[][0em]{Left-aligned labels work best when users are likely to scan both labels and  controls vertically and the labels do not vary much in length. Downsides to this format are that the gaps between the labels and controls can become large if the labels vary in length, the labels and and controls look like two individual columns---giving a cluttered appearance, and it takes a lot of horizontal space if there are long labels or controls. }
  \label{fig:left_align}
\end{figure}

\begin{figure}%
  \includegraphics[width=\textwidth]{../figures/right_align.png}
  \caption[][0em]{Right-aligned labels with left-justified controls work well when users are reading more than they are scanning , such as in a form. In contrast to left-aligned labels , this format's consistent space between the labels and controls makes the pair easier to read and it looks like one column instead of two. }
  \label{fig:right_align}
\end{figure}

Left-aligned labels with ragged controls work best when users are likely to scan vertically to find specific labels but not likely to scan the controls vertically, such as when displaying a set of properties. 

For smartphones, top-aligned labels work best because users can stay oriented when the screen is zoomed and the vertical format works well with vertical scrolling.


\section{Affordances} % (fold)
\label{sec:affordances}
\marginnote{Source: \url{https://developers.google.com/web/fundamentals/accessibility/semantics-builtin/}, 
\url{http://www.satukyrolainen.com/affordances/}
}


The concept of affordance came originally from perception psychologist James Gibson in 1979, and the idea was later modified and made more popular by Don Norman in his book ``The Design of Everyday Things'' from 1988.

When we use a man-made tool or device, we typically look to its form and design to give us an idea of what it does and how it works. An affordance is any object that offers, or affords, its user the opportunity to perform an action. The better the affordance is designed, the more obvious or intuitive its use.

So, affordances are \textbf{clues} about how an object should be used, typically provided by the object itself or its context. For example, even if you've never seen a coffee mug before, its use is fairly natural. The handle is shaped for easy grasping and the vessel has a large opening at the top with an empty well inside.

\begin{figure}%
  \includegraphics[width=\textwidth]{../figures/teapot.png}
  \caption[][0em]{The design of the teapot clearly indicates its intended usage}
  \label{fig:teapot}
\end{figure}

The teapot can be used correctly even without having seen it before because the affordance is similar to those you have seen on many other objects---watering pots, beverage pitchers, coffee mugs, and so on. You probably could pick up the pot by the spout, but your \textbf{experience with similar affordances} tells you the handle is the better option.

In graphical user interfaces, \textbf{affordances represent actions} we can take, but they can be ambiguous because there is no physical object to interact with. GUI affordances are thus specifically designed to be unambiguous: buttons, check boxes, and scroll bars are meant to convey their usage with as little training as possible.

For example, you might paraphrase the use of some common form elements (affordances) like this:
\begin{itemize}
	\item Radio buttons --- \emph{``I can choose one of these options.''}
	\item Check box --- \emph{``I can choose `yes' or `no' to this option.''}
	\item Text field --- \emph{``I can type something into this area.''}
	\item Dropdown --- \emph{``I can open this element to display my options.''}
\end{itemize}

You are able to draw conclusions about these elements only because you can see them. Naturally, someone who can't see the visual clues provided by an element can't comprehend its meaning or intuitively grasp the value of the affordance. So we must make sure that the information is expressed flexibly enough to be accessed by assistive technology that can construct an alternative interface to suit its user's needs.

This non-visual exposure of an affordance's use is called its semantics.

With respect to visual design, affordances are the visual properties of a UI element that indicate how to perform an interaction. 
Standard controls all have affordances.
Push buttons for example have frames and shade that visually suggest they can be pushed, i.e., clicked or touched. 

[TODO: Add examples about affordance ]

The consistent use of affordances is essential for providing an intuitive UI. Affordances often introduce visual clutter; therefore, affordances should be removed when the interaction is already clear of the affordance is redundant. 

Challenge: How do you operate the following thing:
\begin{figure}%
  \includegraphics[width=\textwidth]{../figures/faucet-with-false-affrodance.jpg}
  \caption[][0em]{Try guessing how to use this ``thing''... (the answer will be provided in the lecture) \par Picture source: \url{http://www.satukyrolainen.com/affordances/}}
  \label{fig:faucet}
\end{figure}





